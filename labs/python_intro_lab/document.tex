\documentclass[12pt, a4paper]{article}
\usepackage[utf8]{inputenc}
\usepackage{geometry}
\usepackage{hyperref}
\usepackage{listings}
\usepackage{xcolor}
\usepackage{titlesec}
\usepackage{fancyhdr}

% Page Setup
\geometry{margin=1in}
\pagestyle{fancy}
\fancyhf{}
\lhead{Python Introduction Lab}
\rhead{CS203 STT AI}
\rfoot{\thepage}

% Code Listing Style
\definecolor{codegreen}{rgb}{0,0.6,0}
\definecolor{codegray}{rgb}{0.5,0.5,0.5}
\definecolor{codepurple}{rgb}{0.58,0,0.82}
\definecolor{backcolour}{rgb}{0.95,0.95,0.92}

\lstdefinestyle{mystyle}{
    backgroundcolor=\color{backcolour},   
    commentstyle=\color{codegreen},
    keywordstyle=\color{magenta},
    numberstyle=\tiny\color{codegray},
    stringstyle=\color{codepurple},
    basicstyle=\ttfamily\footnotesize,
    breakatwhitespace=false,         
    breaklines=true,                 
    captionpos=b,                    
    keepspaces=true,                 
    numbers=left,                    
    numbersep=5pt,                  
    showspaces=false,                
    showstringspaces=false,
    showtabs=false,                  
    tabsize=2
}

\lstset{style=mystyle}

\title{\textbf{Lab: Transitioning from Colab to Local Python Development}}
\author{Introduction to Python}
\date{}

\begin{document}

\maketitle

\section{Overview}
In this lab, we will move beyond the browser-based Google Colab environment and learn how to run Python code on your own computer.

\textbf{Duration:} 1 - 1.5 Hours

\section{Prerequisites: Installation}
Before we begin, you must have Python installed. 

You should also install a code editor such as \textbf{VS Code} (recommended) or any other suitable editor (e.g., Sublime Text, Atom, PyCharm). VS Code can be downloaded from \url{https://code.visualstudio.com/}.

\subsection{Video Tutorial}
Please watch this video for a step-by-step guide on installing Python for Windows and Mac:
\begin{center}
    \href{https://www.youtube.com/watch?v=YYXdXT2l-Gg}{\textbf{CLICK HERE: Python Tutorial for Beginners - Install and Setup (Corey Schafer)}}
\end{center}

\subsection{Important Step for Windows Users}
When installing Python on Windows, you will see a checkbox at the bottom of the installer that says:
\begin{center}
    \textbf{``Add Python to PATH''}
\end{center}
\textbf{You MUST check this box.} If you miss it, you won't be able to run Python easily from the command prompt.

\section{Topic 1: The Command Line Interface (CLI)}
The terminal (or Command Prompt) is how you interact with your computer's file system textually.

\subsection{Basic Navigation Commands}
Open your terminal (Command Prompt/PowerShell on Windows, Terminal on Mac/Linux) and try these:

\begin{table}[h]
\centering
\begin{tabular}{|l|l|l|l|}
\hline
\textbf{Action} & \textbf{Windows (cmd)} & \textbf{Mac / Linux} & \textbf{Description} \\ \hline
Check Directory & \texttt{cd} or \texttt{chdir} & \texttt{pwd} & Print Working Directory \\ \hline
List Files & \texttt{dir} & \texttt{ls} & List files in current folder \\ \hline
Change Folder & \texttt{cd foldername} & \texttt{cd foldername} & Enter a folder \\ \hline
Go Back & \texttt{cd ..} & \texttt{cd ..} & Go up one folder level \\ \hline
Make Folder & \texttt{mkdir name} & \texttt{mkdir name} & Create a new directory \\ \hline
Clear Screen & \texttt{cls} & \texttt{clear} & Clears text from screen \\ \hline
\end{tabular}
\end{table}

\section{Topic 2: Running Python Scripts Locally}
In Colab, you press "Play". Locally, you tell the Python interpreter to read a text file.

\begin{enumerate}
    \item Create a folder named \texttt{lab\_demo}.
    \item Inside, create a file named \texttt{hello.py} with the content: \texttt{print("Hello World")}.
    \item Open your terminal, navigate to the folder using \texttt{cd}.
    \item Run the command:
    \begin{itemize}
        \item \textbf{Windows:} \texttt{python hello.py}
        \item \textbf{Mac/Linux:} \texttt{python3 hello.py} (sometimes just \texttt{python})
    \end{itemize}
\end{enumerate}

\section{Topic 3: Imports and Modular Code}
Real software is split across multiple files. 

\subsection{The \texttt{import} system}
If you have \texttt{helper.py} and \texttt{main.py} in the same folder:

\textbf{helper.py:}
\begin{lstlisting}[language=Python]
def greet(name):
    return f"Hello, {name}"
\end{lstlisting}

\textbf{main.py:}
\begin{lstlisting}[language=Python]
from helper import greet

print(greet("Student"))
\end{lstlisting}
Run \texttt{main.py} to see how imports work.

\section{Topic 4: Virtual Environments (Venv)}
Virtual environments isolate your project. One project might need \texttt{pandas} version 1.0, another needs version 2.0. Venv keeps them separate.

\subsection{Creating and Activating}
Run these commands in your project folder:

\textbf{1. Create the environment (named 'venv'):}
\begin{itemize}
    \item Win/Mac/Linux: \texttt{python -m venv venv} (or \texttt{python3 ...})
\end{itemize}

\textbf{2. Activate the environment:}
\begin{itemize}
    \item \textbf{Windows (Command Prompt):} \texttt{venv\textbackslash Scripts\textbackslash activate}
    \item \textbf{Windows (PowerShell):} \texttt{venv\textbackslash Scripts\textbackslash Activate.ps1}
    \item \textbf{Mac/Linux:} \texttt{source venv/bin/activate}
\end{itemize}
When active, your terminal prompt will show \texttt{(venv)}.

\section{Topic 5: Installing Packages (Pip)}
\texttt{pip} is the package installer for Python.

\begin{itemize}
    \item \textbf{Install a package:} \texttt{pip install requests}
    \item \textbf{List installed packages:} \texttt{pip list}
    \item \textbf{Save requirements:} \texttt{pip freeze > requirements.txt}
    \item \textbf{Install from file:} \texttt{pip install -r requirements.txt}
\end{itemize}

\section{Lab Activity}
Download the provided \texttt{lab\_demo} folder. It contains:
\begin{itemize}
    \item \texttt{main.py}: The entry point script.
    \item \texttt{my\_module.py}: A helper module.
    \item \texttt{requirements.txt}: List of dependencies.
\end{itemize}

\textbf{Task:}
\begin{enumerate}
    \item Navigate to \texttt{lab\_demo} in your terminal.
    \item Create a virtual environment: \texttt{python -m venv venv}.
    \item Activate it.
    \item Install dependencies: \texttt{pip install -r requirements.txt}.
    \item Run the code: \texttt{python main.py "Your Name"}.
\end{enumerate}

\end{document}
